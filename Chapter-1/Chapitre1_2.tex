\documentclass{book}
\usepackage[allcolors=true]{hyperref}
\usepackage{graphicx}
\usepackage{amsmath,amsfonts,amsthm,amssymb}
\usepackage{physics}
\usepackage{wasysym}

\newcommand\bg[2]{
\begin{center}
\fcolorbox{white}{BGgris}{\parbox{.9\linewidth}{\begin{large} \textit{#1} \end{large} \\
#2 }}
\end{center}}

\begin{document}

\chapter{Principe d'incertitude d'Heisenberg}\label{chap:chap1}

\section{Introduction}

Le premier chapitre d'un cours de mécanique quantique couvre souvent l'histoire du développement de la théorie; ou bien explique la "vieille mécanique quantique", c'est à dire les résultats théorique comme le modèle de Bohr qui précèdent le véritable développement de la théorie à partir de 1925.

Ici nous allons plustot aborder une approche similaire à l'analyse dimensionnelle. Nous allons partir de la relation d'incertitude d'Heisenberg et l'utiliser pour dériver comment elle peut être utilisée pour expliquer un grand nombre d'ordres de grandeur en physique.

\section{La relation d'Incertitude}

En mécanique classique une particule est charactérisée par des grandeurs (position, vitesse) qui prennent des valeurs bien définie à chaque instant $x(t)$, $v(t)$.
 
Par contre en mécanique quantique, ces grandeurs physique (position, vitesse, moment angulaire, etc. ) sont incertaines et ne peuvent pas être connues avec une précision absolue. Cette incertitude est inhérente à la mécanique quantique. 
Une manière de représenter cela est d'introduire le principe d'incertitude de Heisenberg. 
Concrètement, ce principe nous dit que la position $x$ et l'impulsion $p$ d'une particule sont toutes deux incertaines, et que ces incertitudes sont bornées inférieurement par 
\begin{equation}
\label{Heinsenberg}
\Delta x \Delta p \geq \frac{\hbar}{2}\ .
\end{equation}
où $\hbar = \frac{h}{2\pi}$ avec $h = 6,626 \times 10^{-34}$ J.s la constante de Planck. 

L'origine de cette inégalité est la dualité onde-particule. A la particule nous associons une onde. Or une onde ne peut jamais être localisée parfaitement. D'ou l'incertitude $\Delta x$. Plus précisément on va  attribuer système quantique une fonction d'onde $\psi(\Vec{x},t)$, de sorte que son module au carré sera interprété comme une densité de probabilité de présence à un endroit $\Vec{x}$ à un instant $t$. $\Delta x$ est l'écart type de $x$ pour cette densité de probabilité. 

Et de manière similaire une onde localisée (un paquet d'onde) ne peut avoir un nombre d'onde $k$  parfaitement défini (sinon c'est une onde plane, qui n'est pas physique). L'incertitude en $k$ se traduit en une incertitude en $p$. En effet nous avons la relation de de Broglie
\begin{equation}
    \label{Broglie}
    \lambda = \frac{h}{p}
    \end{equation}
qui lie la longueur d'onde, et donc le nombre d'onde, à l'impulsion.

Il existe d'ailleurs une relation d'incertitude pour les paquets d'onde $\Delta x \Delta k \geq \frac{1}{2}$ qui est vraie pour toutes les ondes. Cette relation d'incertitude pour les ondes prends la forme \eqref{Heinsenberg} dans le cas de la mécanique quantique.

Ultérieurement dans ce cours, nous démontrerons le résultat suivant.
Si l'on mesure la position $x$ de la particule, cette mesure donnera un résultat aléatoire dont l'écart type est $\Delta x$. Et si l'on mesure l'impulsion $p$ de la particule, cette mesure donnera un résultat aléatoire dont l'écart type est $\Delta p$. (Il faut noter que ces deux mesures sont mutuellement exclusive: soit on fait l'une, soit on fait l'autre). Ces deux incertitudes sont liées par \eqref{Heinsenberg} .

(Une autre interprétation de la relation d'incertitude d'Heisenberg est la suivante. Si l'on mesure la position $x$ avec une précision $\Delta x$, alors l'incertitude sur l'impulsion augmentera de $\Delta p$. Et de manière  similaire, si l'on mesure la l'impulsion $p$ avec une précision $\Delta p$, alors l'incertitude sur la position augmentera de $\Delta x$. Cette deuxièrme interprétation est nettement plus difficile à démontrer).

Remarque: de ce qu'il précède, la notion de mesure a clairement un rôle important en mécanique quantique (tandis qu'elle ne joue essentiellement aucun role en mécanique classique). Nous aborderons la mesure en détail plus loin dans le cours.

Nous avons donc que pour tout état quantique, $x$ et $p$ sont incertains : leurs incertitude obéissent à la relation \eqref{Heinsenberg}. \\


\subsection{Relations de de Broglie}
L'incertitude sur l'impulsion et sur la position est liée au caractère \textbf{probabiliste} de la Mécanique Quantique : chaque résultat d'une mesure est aléatoire. Cela fera l'objet de la discussion d'un postulat très important en MQ. 


Elle est également liée à la longueur d'onde de de Broglie. En effet, la relation qui fait intervenir cette longueur d'onde résulte du fait que l'on associe une onde à une particule. C'est ce qu'on appelle la \textit{dualité onde-corpuscule}. 
Il y a par conséquent une correspondance entre les caractéristiques d'une onde (fréquence, longueur d'onde), et les grandeurs physique qui caractérisent une pariticule.
Ces relations de de Broglie sont:
\begin{itemize}
    \item $E = h \nu$ (Energie d'un photon)\\
    \item $\lambda = \frac{h}{p}$ (Longueur d'onde de de Broglie)
\end{itemize}

Notons que ces deux relations restent valables dans le cas relativiste (ce que nous utiliserons dans la suite de ce chapitre).
Rappelons rapidement l'expression de l'énergie d'une particule, de masse au repos $m_0$, d'impulsion $p$, dans le cas relativiste : $E = \sqrt{p^2c^2 + m_0^2c^4}$

 Notons que la longueur d'onde correspondant à une particule de masse $m$ et de vitesse $v$ est d'autant plus grande que $m$ et $v$ sont petits. Ainsi, pour des masses grandes et des vitesses grandes, la longueur d'onde associée devient négligeable (par rapport aux autres distances charactéristiques du système). C'est dans cette limite que nous  retrouvons la physique classique. Autrement dit, les propriétés ondulatoires de la matière est impossible à mettre en évidence dans le domaine macroscopique.  
 
(Note: il y a une limite similaire en optique. La propagation de la lumière est décrite par les équations de Maxwell, qui sont des équations d'onde. Mais si la longueur d'onde est petite par rapport à toutes les autres longueurs charactéristiques, alors on peut traiter la lumière comme des particules qui suivent des trajectoirs qu'on appelle rayons lumineux. C'est l'optique géometrique).
 


\section{Applications du Principe d'Incertitude}

Nous allons montrer que la relation d'incertitude d'Heisenberg permet de déterminer l'ordre de grandeur des effets quantiques dans de nombreux systèmes. 

L'image qu'il faut avoir en tête pour la suite de ce chapitre est la suivante. La particule est délocalisée. Elle n'est pas à une position précise, mais simultanément à un ensemble de positions charactérisée par une étendue (une incertitude) $\Delta x$. De même la particule n'a pas une impulsion précise, mais a simultanément une ensemble d'impulsion charactérisée par une incertitude $\Delta p$. Ces deux incertitudes sont liées par \eqref{Heinsenberg}.


Notons  que ce qui importe ici sera de déterminer des ordres de grandeurs plutôt que des valeurs précises. Nous ne prêterons dès lors pas  attention au facteur $\frac{1}{2}$ qui apparaît dans la relation d'Heisenberg $\ref{Heinsenberg}$.




\subsection{Application à l'Oscillateur Harmonique }
\label{Application à l'Oscillateur Harmonique Quantique}

Rappelons avant tout qu'en général, on définit un \textbf{oscillateur} comme un système qui évolue périodiquement dans le temps, et l'on dit qu'il est en plus harmonique lorsque cette évolution est une fonction sinusoïdale\footnote{de fréquence et d'amplitude constantes.}. \\
Nous allons à nouveau faire appel au principe d'incertitude afin de déterminer l'énergie de l'oscillateur harmonique. 

%\bg{
Remarque: l'importance de l'oscillateur harmonique quantique.%}{
Il est important de savoir que l'oscillateur harmonique joue un rôle fondamentale en physique ; il est à la base d'un très grand nombre d'applications et de théorie.
Par exemple le mouvement des atomes dans un solide, les vibrations au sein des molécules, l'optique quantique. Vous le rencontrerez certainement de nombreuses fois dans le reste de votre parcours en physique!
%}

Un  exemple d'application en MQ serait d'utiliser l'oscillateur harmonique pour décrire le mouvement oscillatoire d'une particule autour de sa position d'équilibre. Nous allons alors procéder comme suit. Ecrivons l'énergie mécanique totale, de manière classique, de l'oscillateur harmonique : 
\begin{equation}
    \label{Energie OHC}
    E = \frac{p^2}{2m} + \frac{kx^2}{2}\ .
\end{equation}

En mécanique classique, l'état d'énergie minimale est $x=0$ et $p=0$, résultant en $E=0$. Mais poser $x=0$ et $p=0$ est en contradiction avec la relation d'incertitude d'Heisenberg. En mécanique quantique, l'état d'énergie minimale aura une énergie non nulle!

Posons $w = \sqrt{k/m}$, la fréquence angulaire. 
\newline Notons $\Delta x \equiv$ l'étendue de l'oscillateur harmonique dans son état fondamental. 
\newline Notons $\Delta p \equiv$ l'incertitude de l'impulsion de l'oscillateur harmonique dans son état fondamental. 
\newline Nous pouvons dés lors estimer l'énergie de l'état fondamental par
\begin{equation}
    \label{Energie OHC}
    E = \frac{\Delta p^2}{2m} + \frac{k \Delta x^2}{2}\ .
\end{equation}
\newline Supposons $\Delta x$ fixé. Utilisons la relation d'Heisenberg, pour rempacer  $\Delta p$ par $\Delta p \approx \hbar/\Delta x$. (Clairement, il vaut mieux que $\Delta p$ soit petit si l'on veut que l'énergie soit faible. Donc nous supposons que la relations d'incertitude est approximativement saturée).
\newline En partant de ces hypothèses, on exprime l'énergie de l'état fondamental en fonction de $\Delta x$ par: 
\begin{equation}
E \left(\Delta x \right) \approx \frac{\hbar^2}{2m\Delta x^2} + \frac{1}{2} k \Delta x^2
\end{equation}
\newline  Ensuite, nous pouvons minimiser $E \left(\Delta x \right) $ par rapport à $\Delta x$. Posons
\begin{equation}
\frac{ \partial E \left(\Delta x \right) }{\partial \Delta x} =0
\end{equation}
et résolvons cette équation pour $\Delta x$.
On trouve que l'énergie est minimale en $$\Delta x \approx \frac{\hbar^{1/2}}{\left(mk\right)^{1/4}}\; .$$ 
En réinjectant l'expression de l'étendue $\Delta x$ pour laquelle l'énergie est minimale, dans l'expression de l'énergie de l'oscillateur même, on trouve ainsi l'énergie minimale : 
\begin{align}
\label{Energie OH}
E_{min} &\approx \hbar \sqrt{\frac{k}{m}} \notag \\
& \approx \hbar \omega 
\end{align}

\paragraph{} De cette expression, nous pouvons en tirer une conclusion très importante qui est que l'énergie à l'état fondamental de l'oscillateur harmonique est non nul, contrairement aux résultats prédits par la physique classique. 

\paragraph{} Lorsque nous aurons les outils nécessaires pour aborder l'oscillateur harmonique quantique (OHQ) en détail, nous verrons que les valeurs que peuvent prendre l'énergie de l'OHQ sont quantifiés et vérifient la relation suivante :
\begin{align}
\label{Energie quantifiée OHQ}
E_n = \hbar \omega \left( n + \frac{1}{2} \right) \mbox{ ; $n \in \mathbb{N} $}
\end{align}
De cette expression \eqref{Energie quantifiée OHQ}, nous observons que : 
\begin{itemize}%[label=\textbullet]
    \item D'une part, le spectre des énergies est discrets ; ainsi une certaine quantité d'énergie définie doit être fournie au système afin de passer d'un état $n$ à un état supérieur (et à l'inverse, le système doit "perdre" une certaine quantité d'énergie pour passer à un état inférieur) ; \\
    \item D'autre part, l'énergie fondamentale, pour $n=0$, vaut $$ E_0 = \frac{1}{2} \hbar \omega $$
    On l'appelle également \textbf{Energie de point zéro} (venant de l'anglais, "\textit{zero point energy}", et est bien non nulle. 
    \newline L'expression que l'on a obtenu en utilisant le principe d'incertitude concorde donc bien, en terme d'ordre de grandeur (puisque l'on avait la même expression à un facteur $\frac{1}{2}$ près). 
\end{itemize} 


\subsection{Application à l'atome d'Hydrogène}
Le but ici est de comprendre la stabilité des atomes et de retrouver l'ordre de grandeur de l'énergie de l'atome d'hydrogène dans son état fondamental. \\

En effet, l'énergie mécanique totale d'un atome d'hydrogène peut s'écrire comme la somme de son énergie cinétique (mouvement de l'électron) et de son énergie potentiel (coulombien) : 
\begin{align}
\label{Energie hydrogène}
E &= \frac{p^2}{2m} + \frac{q^2}{4\pi \epsilon_0 r} \notag \\
&= \frac{p^2}{2m} - \frac{e^2}{4 \pi \epsilon_0 r} \notag \\
\implies E &= \frac{p^2}{2m} - \frac{\mathcal{E}^2}{r}
\end{align}
où $\mathcal{E}^2 = \frac{e^2}{4 \pi \epsilon_0}$.\\

Supposons que l'électron de l'atome est confiné dans une zone de rayon $\Delta r$ autour de l'origine du à l'attraction coulombienne. Nous voulons déterminer le rayon $\Delta r$ qui minimise l'énergie \eqref{Energie hydrogène}. Classiquement, c'est évidement $\Delta r = 0$ et $E=-\infty$. Mais quantiquement, il y a un état d'énergie minimal finie, corresondant à un $\Delta r \neq 0$. 

Utilisons la relation d'Heisenberg \eqref{Heinsenberg}. Nous avons que $\Delta p \approx \frac{\hbar}{\Delta r}$. (Plus précisément, la relation d'incertitude est vraie pour chaque composante $\Delta x \Delta p_x \geq \frac{\hbar}{2}$, etc... Si la particule est localisée dans une zone sphérique de rayon $\Delta r$, alors $\Delta x \approx \Delta y \approx \Delta z \approx \Delta r$.  Et $\Delta p^2 \approx \Delta p_x^2 + \Delta p_y^2 + \Delta p_z^2 \approx \hbar^2 / \Delta r^2$. Ce qui donne $\Delta p \approx \frac{\hbar}{\Delta r}$).


Ainsi : 
\begin{equation}
\label{Energie hydrogene approx}
E(\Delta r) \approx \frac{\hbar^2}{2m \Delta r^2} - \frac{\mathcal{E}^2}{\Delta r}
\end{equation}
\paragraph{} A présent, déterminons la valeur de $\Delta r$ pour lesquelles l'énergie est minimale. Pour ce faire, calculons les racines de la dérivée de l'énergie :
\begin{align}
\label{rayon d'energie minimale}
\frac{dE}{d\Delta r} &= \frac{d}{d\Delta r} \left( \frac{\hbar^2}{2m\Delta r^2} - \frac{\mathcal{E}^2}{\Delta r} \right) \notag \\
&= - \frac{2\hbar^2}{2m\Delta r^3} + \frac{\mathcal{E}^2}{\Delta r^2} \notag \\
&= \frac{1}{\Delta r^2} \left( \mathcal{E}^2 - \frac{\hbar^2}{m \Delta r} \right) = 0 \notag \\
\iff \Delta r&= \frac{\hbar^2}{m\mathcal{E}^2}
\end{align}

\paragraph{} Nous avons donc trouvé un rayon $\Delta r$ pour lequel l'énergie de l'atome d'Hydrogène est minimale, et cette énergie minimale vaut alors, en combinant \ref{rayon d'energie minimale} et \ref{Energie hydrogene approx}: \begin{equation}
    \label{Energie minimale}
    E_{min} = - \frac{m\mathcal{E}^4}{2 \hbar^2}
\end{equation}

En fait, le rayon $\Delta r$ donné par la relation \ref{rayon d'energie minimale} et l'énergie minimale qui lui est associée en \ref{Energie minimale} correspondent respectivement au rayon du modèle de Bohr $a_0$ (où $a_0 = \frac{\hbar^2}{m\mathcal{E}^2} \equiv$ la distance (moyenne) séparant, dans l'atome d'hydrogène, le proton de l'électron) et à l'énergie de Rydberg $R_y$ (où $R_y = \frac{m\mathcal{E}^4}{2\hbar^2} \equiv$ l'énergie de liaison de l'atome d'Hydrogène). \\

La mécanique quantique prédit que l'énergie ne peut prendre que des valeurs discètes. La résolution de l'équation de Schrödinger pour l'atome d'Hydrogène (qui ne sera pas faite dans ce cours) montre que les énergies des états liés (états d'énergie négative) est donnée par 
\begin{equation}
\label{Etats liés}
E_n = - R_y \frac{1}{n^2}
\end{equation}
où $n$ = 1,2,3, ...




\subsection{Principe d'incertitude dans le cas relativiste}
\label{Principe incertitude relativiste}
\paragraph{} Il se trouve que la relation d'Heisenberg est encore valable dans un cadre relativiste. Or nous savons que  l'impulsion et l'énergie d'une particule libre relativiste est exprimée par : 
\begin{equation}
    \label{Energie relativiste particule libre}
    E^2 = m^2 c^4 + p^2 c^2\ .
\end{equation}

Considérons un électron que l'on confine dans une zone de taille $\Delta r$.
Son impulsion a une incertitude $\Delta p \approx \hbar / \Delta r$.
En utilisant
\eqref{Energie relativiste particule libre} nous avons que
$E = \sqrt{ m^2 c^4 + \Delta p^2 c^2}$ ce qui nous donne l'incertitude sur l'énergie due à l'incertitude sur la position (en supposant que l'impulsion moyenne est nulle).

Si l'on réduit $\Delta r$, $\Delta p$ augmente. 
Si $\Delta r$ est suffisement petit, $\Delta x \approx \frac{\hbar}{mc}$, alors
$\Delta p \geq mc$. La particule deviens relativiste, simplement par le fait qu'elle est confinée. Dans ce cas, l'incertitude sur l'énergie vérifie alors la relation 
 $ \Delta E \geq mc^2$.
 Dans le cas où elle excède $2mc^2$, il y a suffisamment d'énergie que pour donner naissance à une paire : l'électron avec son positron associé. C'est ainsi que la notion de \textit{"une particule"} perd son sens dans la mécanique quantique relativiste. On ne peut pas parler d'"une particule" car l'incertitude d'Heisenberg fait que le nombre de particules peut devenir incertain.



\paragraph{} Fixons à présent $\Delta p = mc$, de sorte que, par la relation d'Heisenberg, on est dans le cas relativisite. On a que $\Delta x \geq \frac{\hbar}{mc}$ (au facteur numérique près). Le membre de droite porte un nom, c'est la longueur d'onde de Compton $\lambda_C$ : \begin{equation}
    \label{longueur d'onde de compton}
    \lambda_C = \frac{h}{mc}
\end{equation}
\newline  C'est la longueur d'onde de Broglie à laquelle les effets relativistes deviennent important.



\subsection{La Masse de Planck}
Considérons une particule de masse M. Nous pouvons associer 2 longueurs à cette particule.
\begin{enumerate}
\item \textbf{Première longueur caractéristique :} La longueur de Compton : $\lambda_C = \frac{\hbar}{Mc}$.
\item \textbf{Seconde longueur caractéristique.} Le rayon de Schwarzschild : $R_s = \frac{2GM}{c^2}$ 
avec $G \equiv$ la constante universelle de gravitation

Ce rayon caractérise le fait que si un objet de masse M est confiné à l'intérieur d'une sphère de rayon plus petit ou égal à $R_S$, alors cet objet est en fait un trou noir, dont l'horizon est la sphère de rayon $R_S$. 
 Aucun objet, pas même la lumière, ne peut s'échapper d'un trou noir. (En mécanique classique, $R_S$ est le rayon tel que la vitesse de libération est $c$). 
\end{enumerate}


Considérons la masse pour laquelle 
la longueur d'onde de Compton réduite (c'est-à-dire la longueur d'onde de Compton en utilisant la constante de Planck réduite $\lambda_C = \hbar/(Mc)$)
est égale à la moitié du Rayon de Schwarzschild: 
\begin{align}
\label{Masse de Planck}
\frac{\hbar}{Mc} &= \frac{GM}{c^2} \notag \\ 
\iff M & \equiv m_p \mbox{ (\textbf{Masse de Planck})} = \sqrt{\frac{\hbar c}{G}}
\end{align}

La masse de Planck joue un rôle très important en physique.
Pour une particule / un objet de masse plus petite que $m_P$, nous utiliserons la mécanique quantique (relativiste) pour décrire cette particule.
Pour une particule / un object de masse plus grande que $m_P$, nous utiliserons la relativité générale.
Il n'y a aujourd'hui aucune théorie satisfaisante pour décrire une particule de masse approximativement égale à $m_p$, car nous n'avons pas de théorie de la gravitation quantique.

Notons que Planck, dans son article fondateur sur la mécanique quantique, 
s'est rendu compte qu'il était possible de faire un jeu d'unités fondamentales en combinant les grandeurs
$\hbar$, $c$ et $G$. En combinant ces grandeurs, on peut également faire une unité de distance (la longueur de Planck) et une unité de temps (le temps de Planck).



%\paragraph{} \textit{Remarque :} Notons que nous avons fixé la taille de l'objet avant de comparer sa masse afin de raisonner en terme de \textbf{densité} d'énergie. En effet, nous ne pourrions pas conclure que la Terre soit un trou noir, car bien que sa masse soit bien plus importante que la masse de Planck, sa densité d'énergie est faible.

\subsection{Application à la Masse des étoiles}
Considérons N atomes d'hydrogène dans une boule de rayon R, soumis à l'attraction gravitationelle. Nous supposerons les atomes à température nulle (plus précisément nous supposerons qu'on peut négliger l'énergie cinétique due à l'agitation thermique par rapport à l'énergie cinétique due à la relation d'incertitude de Heisenberg). Les astronomes appèlent cela de la matière dégénérée.

Notons les approximations suivantes : 
\begin{itemize}%[label = \textbullet]
    \item Ordre de grandeur du volume par atome $\approx \frac{R^3}{N}$ ;\\
    \item On en tire le rayon de confinement de chaque électron et proton : $\Delta x_e = \frac{R}{N^{1/3}} = \Delta x_p$ ; \\
    \item Et enfin, on en déduit, par la relation d'Heisenberg, les incertitudes de leur impulsions : $\Delta p_e = \frac{\hbar N^{1/3}}{R} = \Delta p_p$
\end{itemize} 

Rappelons que la masse du proton $m_p$ est approximativement $2000$ fois la masse de l'électron $m_e$. Nous utiliserons donc l'approximation $m_p \gg m_e$.

\paragraph{} De ces relations, écrivons l'énergie totale de l'étoile, en exprimant d'une part l'énergie cinétique des électrons et protons, et d'autre part l'énergie potentielle qui est due au fait que les particules sont soumises à une attraction gravitationnelle : \\

\begin{itemize}%[label = \textbullet]
\item Energie cinétique (non-relativiste) : $N \left( m_e c^2 + \frac{\Delta P_e^2}{2 m_e} + m_p c^2 + \frac{\Delta P_p^2}{2 m_p} \right)$ \\
Nous pouvons négliger le dernier terme de cette expression du fait que l'on peut considérer que $m_p >> m_e$  \\
\item Energie potentielle de gravitation : $- G \frac{(Nm_p)^2}{R}$  (où nous négigeons la masse des électrons).\\
%Pas encore tout à fait clair pourquoi
\end{itemize}

Dès lors, l'énergie totale sera donnée par :
\begin{equation}
E(R) \approx - G \frac{(Nm_p)^2}{R} + Nm_e c^2 + Nm_p c^2 + \frac{N}{2m_e} \frac{\hbar^2 N^{2/3}}{R^2}
\end{equation}

Nous pouvons à présent déterminer un rayon $R^*$ pour lequel l'énergie est minimale, en extrémisant cette fonction de l'énergie $E$ dépendant de $R$ : 
\begin{align}
\label{rayon d'energie minimale étoile}
    &\frac{d}{dR} E(R) = 0 \notag \\
    \iff &R^* = \frac{\hbar^2}{Gm_p^2m_eN^{1/3}}
\end{align}
On remarque que le rayon $R^*$ diminue lorsque le nombre de particules N augmente. \\

Or nous avons vu dans la section \ref{Principe incertitude relativiste} qui traite du principe d'incertitude dans le cadre relativiste, qu'il est possible d'avoir des particules relativistes ($\Delta p \approx mc$) lorsqu'elles sont confinées dans une région suffisamment petites. Il est alors naturelle de se poser la question ; à partir de combien de particules N est-ce que les électrons deviennent relativistes ? Autrement dit, le rayon étant relié à $N$, nous allons simplement chercher le nombre $N$, tel que le rayon $R^*$ nous permet d'obtenir une incertitude de l'impulsion de l'ordre de $mc$.  \\

En effet nous avons que : 

\begin{align*}
    \left\lbrace
\begin{array}{ccc}
    \Delta p_e = cm_e \\
    \Delta p_e = \frac{\hbar N^{1/3}}{R^*} = \frac{N^{2/3} G m_e m^2_p}{\hbar}
\end{array}\right.
\end{align*}

\begin{align}
    \implies cm_e &= \frac{N^{2/3} G m_e m^2_p}{\hbar} \notag \\
    \iff N^{2/3} &= \frac{\hbar c}{G}\frac{1}{m^2_p} = \frac{M^2_p}{m^2_p} \notag \\
    \iff N &= \left( \frac{M_p}{m_p} \right)^3
\end{align}

Remarquons que nous n'avons considéré que l'impulsion des électrons. Vu la masse beaucoup plus grande des protons, quand $\Delta p = cm_e$ les électrons sont relativiste, mais pas les protons.

Notons maintenant que l'échelle d'énergie des réactions nucléaires 
\begin{align}
e^- + p^+ &\longrightarrow n \notag \\
n + p^+ &\longrightarrow d^+ \notag \\
d^+ + d^+ &\longrightarrow He^{++} \notag
\end{align}
où $d$ signifie ici deutérium,
est de l'ordre de grandeur de l'énergie de masse de l'électron. (En particulier la première des ces réactions est endothermique, les deux suivantes sont exothermiques).


Nous pouvons donc imaginer la situation suivante. Prenons $N$ atomes d'Hydrogène autogravitant, et augmentons graduellement $N$ tout en maintenant la température nulle (ou très faible). C'est possible jusqu'à ce que $N$ atteigne approximativement $N\approx  \left( \frac{M_p}{m_p} \right)^3$. A ce moment les électrons deviennent relativistes, et peuvent enclancher des réactions
nucléaires de fusion. Il y a libération d'énergie au sein de la boule. Une étoile s'est ainsi formée. \\ 

La masse des étoiles est donc approximativement donnée par $ M \approx Nm_p$ et leur masse par
\begin{equation}
    M \approx m_p \left( \frac{M_p}{m_p} \right)^3\ .
\end{equation}

En terme de valeur, cela donne environ : 
\begin{align*}
    \left\lbrace
\begin{array}{ccc}
M_p \approx 10^{19} \mbox{ GeV.c$^{-2}$} \\
m_p \approx 1 \mbox{ GeV.c$^{-2}$}
\end{array}\right.
\end{align*}
$\implies  N \approx 10^{57}$ (et la valeur exacte pour notre soleil est de $1,04 \times 10^{57}$) ; \\
Ainsi, la masse d'une étoile est environ de l'ordre de $M \approx 10^{30}$ kg. \\

%\bg{
Remarque :% }{
\begin{itemize}%[label = \textbullet]
    \item Pour notre soleil nous avons $M_{\astrosun} \approx 2 \times 10^{30}$ kg);\\
    \item Les plus petites étoiles ont une masse $M \approx 0,08 M_{\astrosun}$ ; \\
    \item Les plus grandes ont une masse $M \approx 100 M_{\astrosun}$ ; \\
    \item 
  Le rayon $R$ des  étoiles est beaucoup plus grand que le rayon $R^*$ calculé plus haut. En effet dans les étoiles, une fois que les réactions thermonucléaires commencent, l'intérieur de l'étoile chauffe, la pression augmente, la masse volumique diminue, et le rayon devient (beaucoup) plus grand que $R^*$. C'est le cas par exemple du soleil.
 
 
    
    Il existe des étoiles pour lesquelles $R$ est proche de $R^*$, mais il faudrait plus de connaissances approfondies en astrophysique pour comprendre cela. 
\end{itemize} 
%}

\subsection{La Masse de Chandrasekhar}

Reprenons $E(R)$ pour $N$ atomes d'hydrogènes dans une boule de rayon R et à température nulle, mais tenons maintenant compte d'effets relativistes.
L'expression de l'énergie du système $E(R)$  deviens maintenant : 
\begin{equation}
E(R) \approx - \frac{GN^2 m_p^2}{R} + N m_p c^2 + N \sqrt{m_e^2c^4 + \frac{\hbar^2N^{2/3}}{R^2}c^2} \\
\end{equation}

Selon la valeur de $N$, la forme de cette courbe change complètement, et par conséquent la valeur de $R^*$ qui minise $E(R)$.
En effet, pour $N$ faible, $\lim_{R\to 0} E(R) = + \infty$. Tandis que pour 
$N$ grand, $\lim_{R\to 0} E(R) = - \infty$, et dans ce cas $R^*=0$ (ce que nous interpreterons ci-dessous).

Déterminons la valeur critique  $N = N^*$ où l'on passe d'un comportement à un autre : 
\begin{align}
\lim_{R\to 0} E(R) &= \lim_{R\to 0} \frac{-GN^2m^2_p + RNm_pc^2 + \sqrt{N^2m^2_ec^4R^2 + \hbar^2 N^{2/3} c^2}}{R} \notag \\
&= \lim_{R\to 0} - \frac{GN^2 m_p^2}{R} + \frac{\hbar N^{4/3} c}{R} = 0 \notag \\
\iff N^{2/3} &= \frac{\hbar c}{Gm^2p} \notag \\
\iff N &= \left( \frac{\hbar c}{G} \right)^{3/2} \frac{1}{m_p^3} = \frac{M_{p}^3}{m_p^3} \notag \\
\implies N &= N^* = \left( \frac{M_p}{m_p} \right)^3
\end{align}

% Schéma du Comportement de E(R) 

\paragraph{} Comment pouvons-nous interpréter cette valeur critique ? 

Si $N < N^*$, alors il existe un rayon $R^*>0$ qui minimise l'énergie.


Par contre, si $N > N^*$, alors la boule ne peut pas résister à son attraction gravitationnelle : il y a alors ce qu'on appelle un effondrement gravitationnel, qui fait que la boule se transforme en un trou noir. 

La masse de Chandrasekhar est la  limite à partir de laquelle il peut y avoir un effondrement gravitationnel. Un calcul plus poussé montre que cette limite est approximativement $M_{limit}=1,4 M_{\astrosun}$. 


La masse de Chandrasekhar joue un rôle très important en astrophysique. Voici ce qu'en dit wikipedia (le texte n'est pas très bien écrit)

(\url{https://fr.wikipedia.org/wiki/Masse_de_Chandrasekhar}
)


{ \it
{\bf Naine blanche.}

Une naine blanche, reste d'étoile après que les couches extérieures en ont été soufflées, est un astre formé de matière dégénérée (c'est à dire pour laquelle l'impulsion des électrons est dominé par la contribution de l'incertitude d'Heisenberg), 
abritant une masse de l'ordre de celle du Soleil dans une sphère d'un rayon de l'ordre de celui de la Terre, d'où une masse volumique très élevée. 
Pour une naine blanche composée de carbone et d'oxygène, la masse de Chandrasekhar s'élève à 1,44 masse solaire (soit environ $2,8 \times 10^{30}$ kg). 
À l'issue de l'évolution d'une étoile, une naine blanche a une masse largement inférieure à la masse de Chandrasekhar (moins d'une masse solaire), mais peut accumuler de la matière, ce qui arrive notamment si une autre étoile orbite près d'elle. Sa masse augmentant, son rayon diminue jusqu'à ce que, près de la masse de Chandrasekhar, la contraction gravitationnelle et les épisodes de novae aient suffisamment réchauffé l'intérieur de l'étoile pour que se déclenche la fusion du carbone. Lorsque la réaction commence, l'augmentation de température à l'intérieur de l'astre est isochore à cause de la dégénérescence (indépendance de la température par rapport à la pression: tant que l'impulsion des électrons est dominé par la contribution de l'incertitude d'Heisenberg, la température augmente sans que la pression augmente). 
Comme la réaction est très fortement dépendante de la température, elle s'emballe et l'énergie libérée dépasse l'énergie de liaison gravitationnelle de la naine blanche. Cette dernière devient une supernova de type Ia. A la différence des supernovas de type ll résultant de l'effondrement du cœur d'étoiles massives ayant consommé tous les éléments pouvant fusionner (jusqu'au fer) et générant une étoile à neutrons ou même un trou noir (au-dessus de 3,2 masse solaires résiduelles), les supernova de type la dispersent la totalité de leur masse. Ce sont de super bombes thermonucléaires faisant fusionner en un temps très court le carbone, l'azote et l'oxygène dont elles sont presque exclusivement constituées.

{\bf  Cœur d'une étoile massive}

Les étoiles d'une masse supérieure à 12 masses solaires sont assez chaudes pour que les réactions de fusion thermonucléaire conduisent à la formation de nickel 56 par fusion du silicium. Lorsque le silicium central a complètement été transformé en fer et en nickel 56 il reste un noyau de nickel 56 et de fer inerte entouré par une couche de silicium toujours en réaction. Le noyau inerte se contracte jusqu'à atteindre la pression de dégénérescence des électrons, et constitue un corps dégénéré dont la masse de Chandrasekhar est d'environ 1,4 masse solaire. La fusion du silicium continue d'apporter du nickel, qui se dépose sur le corps et devient dégénéré à son tour. Lorsque la masse accumulée dépasse la masse de Chandrasekhar, l'effondrement gravitationnel se produit et la supernova commence, laissant une étoile à neutrons d'un diamètre de quelques dizaines de kilomètres et contenant au moins 1,5 fois la masse du soleil.
}

\section{Conclusion}

Nous avons vu dans ce chapitre comment la relation d'incertitude d'Heisenberg permet d'expliquer un grand nombre d'ordres de grandeur en physique. En particulier, il nous a permis d'expliquer 
\begin{itemize}
\item l'énergie de point zéro de l'oscillateur harmonique; \\
\item l'énergie de liaison et la taille de l'atome d'Hydrgène;\\
\item la longueur d'onde de Compton - la distance à laquelle la notion d'1 électron n'est plus relevante car le nombre d'électrons devient incertain; \\
\item la masse de Planck - la masse d'une particule élémentaire dont la description devrait nécessairement faire intervenir la gravitation quantique, les effets quantiques et les effets gravitationels étant du même ordre de grandeur; \\
\item la taille d'une masse d'Hydrogène dégénéré (pour laquelle l'impulsion et l'énergie cinétique des électrons due au principe d'incertitude est plus grande que l'impulsion et l'énergie thermique) autogravitante;\\
\item la masse à partir de laquelle cette masse autogravitante va soit enclancher des réaction nucléaires (devenant une étoile), soit s'effondrer en un trou noir (donnant une supernova).
\end{itemize}




\end{document}
